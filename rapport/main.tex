\documentclass[12pt, a4paper]{article}
\usepackage{graphicx} % Required for inserting images
\usepackage{wrapfig}    % for wrapping figures
\usepackage[swedish]{babel} % Gör vårt projekt till svenska
\usepackage{times} % Använder Times-teckensnitt istället för LaTeX standardteckensnitt
\usepackage[letterpaper,top=2cm,bottom=2cm,left=3cm,right=3cm,marginparwidth=1.75cm]{geometry} % Marginalen något mindre

\usepackage{mathtools, hyperref, url}
\usepackage{amsmath, amssymb, svg}
\usepackage{listings}% Code blocks
\usepackage{minted}
\usepackage{siunitx}                % Fysikaliska enheter och siffror
\usepackage{xcolor}
\usepackage[T1]{fontenc}
\usepackage{subcaption}
\usepackage[utf8]{inputenc}
\usepackage{titlesec}
\usepackage{icomma}

\definecolor{codegreen}{rgb}{0,0.6,0}
\definecolor{codegray}{rgb}{0.5,0.5,0.5}
\definecolor{codepurple}{rgb}{0.58,0,0.82}
\definecolor{backcolour}{rgb}{0.95,0.95,0.92}

\lstdefinestyle{mystyle}{
    backgroundcolor=\color{backcolour},   
    commentstyle=\color{codegreen},
    keywordstyle=\color{magenta},
    numberstyle=\tiny\color{codegray},
    stringstyle=\color{codepurple},
    basicstyle=\ttfamily\footnotesize,
    breakatwhitespace=false,         
    breaklines=true,                 
    captionpos=b,                    
    keepspaces=true,                 
    numbers=left,                    
    numbersep=5pt,                  
    showspaces=false,                
    showstringspaces=false,
    showtabs=false,                  
    tabsize=2
}

\lstset{style=mystyle}

\sisetup{locale = DE, tight-spacing=true, group-digits=true}   % För decimal komma
\titleformat*{\section}{\large\bfseries}

%En kort beskrivning av hur man kör koden 
%    (speciellt viktigt om koden består av flera filer)
%En kort beskrivning av vad koden gör och vilka antaganden som gjorts
%Svar på de frågor som ställts i uppgiftstexten.

\title{Beräkningsuppgift 2: Rankinecykel}
\author{
    Alexander Andersson\\
    Teknisk termodynamik 1FA527 63025\\
    Uppsala Universitet
}

\date{\today}

\begin{document}

\maketitle
\vspace{-25pt}


\section{Introduktion}
\vspace{-5pt}
Denna rapport beskriver en lösning till beräkningsuppgift 2. Koden kan köras
genom att antingen anropa Jupyter notebook på filen del1.ipynb och del2.ipynb eller
genom att köra del1.py och del2.py båda metoder ska ge samma resultatat. Om
nödvändigt kan moduler i requirements.txt behöva installeras.
Uppgift 1 handlar om att beräkna den den termiska verkningsgraden i en vanlig Rankinecykel
utifrån några begränsningar. Uppgift 2 handlar om att hitta den bästa verkningsgraden i en Rankinecykel
med matarvattenförvärmning enligt annars samma villkor.
\section{Antaganden}
\vspace{-5pt}
De centrala antaganden som gjorts är att processerna anses ideala, alltså är turbinen och pumparna
isentropa, samt kondensatorn och ångpannan isobara. Dessutom antas att inga yttre förluster sker
utan all energi tillförs i pumparna och ångpannan vilket sedan tas ut i kondensatorn och turbinen.

\section{Beräkningar}
\vspace{-5pt}
I del 1 kunde tillstånd 3 bestämmas helt från initialvärden och det svåra blev
att beräkna trycket i tillstånd 4. Detta kunde göras genom att hitta roten till ekvationen:
\begin{equation}
    s_f + 0,85(s_g-s_f)- s_3 = 0
\end{equation}
Entropierna $s_f$ och $s_g$ beräknades utifrån olika tryck med 0 respektive
100\% ångkvalité, $s_3$ kunde beräknas redan innan. Ur detta kunde ett värde på trycket $P$ hittas.
Därefter kunde resterande storheter beräknas som vanligt i Rankinecykeln.
Det visade sig att den termiska verkningsgraden blev ca 39,8\% vid ett
kondensatortryck på ca 28 kPa.

Del 2 krävde att en liknande ekvation som del 1 löstes med skillnaden att nu erhölls istället
ett system av olika fraktioner $y$ från turbinen som vardera gav olika tryck. Genom att lösa
systemet och sedan hitta det maximala värdet erhölls att den
termiska verkningsgraden i del 2 uppgick maximalt till ca 42,7\%
vid en fraktion $y$ på ca 20,8\%. Trycket på den avtappade ångan uppgick till ca
1,4 MPa.


%\bibliographystyle{IEEEtran}
%\bibliography{references}


\end{document}
